\section{System Architecture}

We like to think of our system as an hourglass. We aim to separate our
hardware components as much as possible from the algorithms we write,
where our car's firmware is the narrow bridge between these two worlds.

\vspace{1em}

Major hardware components include:
\begin{itemize}
  \item Our new Brake-By-Cable system
  \item Water-proof trunk for power distribution systems
  \item DC-DC converters
  \item Cooper Common Microcontroller Nodes
  \item A CAN Bus for node communication
  \item CCMN hats for our brakes, throttle, and encoders
  \item An ODrive for controlling our EPAS
\end{itemize}

Safety devices include:
\begin{itemize}
  \item Hardware E-STOPs around the car and inside the cabin
  \item ATC and glass tube fuses
  \item Circuit breakers
  \item Undervoltage protection
\end{itemize}

Major software modules include:
\begin{itemize}
  \item Shared firmware on all CCMNs
  \item OpenCAN for serialization/deserialization of CAN messages at cyclic
        rates
  \item Velocity and pressure controllers
  \item ROStouCAN for interfacing between nodes on our CAN Bus and our ROS
        nodes
  \item Lane detection algorithm
  \item Stanley controller for trajectory following
  \item A fine-tuned YOLO model for object detection
  \item State machines for state and event transitions based on the output of
        our object detection and lane detection
\end{itemize}

Even though we try to maintain our hourglass shape, it does not detract
from the close collaboration between our subteams. For example, our
Algorithms team needed more accurate braking on our vehicle this year,
so they met with the hardware subteam to make this possible. It was
then the responsibility of the hardware and firmware teams to work
together to ensure our new brake system performed better without
substantially changing how the system worked from the Algorithm
perspective.
