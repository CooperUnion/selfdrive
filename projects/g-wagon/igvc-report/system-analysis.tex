\section{System Analysis}

While testing things individually, it's easy to keep track of changes,
but once integrated into our system, it's easy to lose track of changes
and decisions. One example of this was during the development of the
BBC PCB. We broke the board into four main blocks: motor controller
logic, parking brake logic, pressure sensor data, and limit switches
for the lever's range of motion. However, when testing the entire
board, we found we would flip the polarity of certain pins, such as the
input to the motor, or flip which connectors went to the minimum limit
switch and the maximum limit switch.

The best way to prevent such mistakes is to create a more robust,
generalized system rather than case-by-case. For example, we can ensure
every wire has a proper visible label rather than memorizing its
location. Additionally, we can write out a simple checklist of what to
check before testing something. By creating a more robust system, it is
easy to prevent many of the smaller, simpler mistakes.

\subsection{Critical Hardware Failure}

One hardware failure that would prevent competition success is if an
ESP32S3 were to fail. The ESP32S3, a low-power microcontroller, limits
the amount of current it can supply. We maintain an external 12V
battery pack on the car because even though a CCMN can supply 12V, it's
limited to tens of milliamps. We handle this by using isolated circuits
to turn on/off the circuit i.e. using NMOS or PMOS transistors due to
their high input resistance as a switch or using a relay to keep the
two systems completely isolated.

\subsection{Software Development}

We store all code in a publically hosted monorepo on GitHub. We use Git
as it allows us to effectively work in parallel, while the monorepo
makes integration between various components straightforward. We
utilize GitHub features such as issues and pull requests to keep track
of problems with our code and encourage code review.

\subsection{Testing}

Before we test a particular algorithm or state machine on the car, and
because our team is limited in physical space to which we can test the
car, we test our ideas on smaller platforms, some of which we developed
within Autonomy Lab. Carrie is our designed in-house mini-car robot
platform for testing trajectory-following controllers. We tested and
tuned the Stanley geometric controller for following clothoid paths,
such as a complex path generated for parallel parking. We found the
Stanley controller to be an effective method of calculating a steering
angle for following a path. For testing with simplified dynamics, we
used a differential-drive TurtleBot platform. To initially test and
tune the OpenCV parameters for lane detection, we placed cameras on the
Turtlebot and added a simple controller to follow the lane lines.

\subsection{Physical Testing to Date}

We tested the BBC system by cycling the motor between the two limit
switches. The initial DC motor did not have enough torque and would
stall before reaching the maximum displacement of the brake pedal. We
sourced a more powerful DC motor with a higher stall torque, but this
larger motor can pull up to 25 Amps. To ensure the motor does not pull
too much current we installed a 7A fast-blow fuse on the battery
supplying power to the motor.

When measuring the pressure sensor output from the car, the original
data for the pressure sensor was from 0.7V to 2.0V. While this range of
values is valid for the ADC readings to the ESP32S3 (within 0.15V to
2.45V), there wouldn’t be enough precision to accurately calculate the
pressure percentage. By using a circuit that combines a non-inverting
OpAmp and a subtractor op amp, we simulated a circuit that linearly
remaps the pressure sensor range from 0.7V to 2.0V into 0.236V to
2.302V. In real-life however, we saw that the actual voltage range was
0.153V to 2.102V. This clearly demonstrates an example of when an ideal
simulation is different from a real-world implementation. The reason
for this difference comes from non-ideal components, device variations,
and parasitic capacitance and inductance throughout our circuit in the
op amps and resistors.

We tuned the inner PID loop for BBC using a pressure transducer and
recording the system's response time. We found that a high proportional
and small integral gain provided a fast response time with no
overshoot. When releasing the brake pedal before installing the spring,
the top cable would become loose and often fall off the pulley, causing
the cable to become plastically deformed. After installing the spring
to keep the lever and top cable in tension with the motor and pulley,
we found a great improvement in the response time and reliability of
the system.

For braking, once the inner loop's response was well tuned and we could
set and hold a desired braking percentage within 1 second, we
characterized the relationship between brake percent and deceleration.
First, we throttled the car to 10 mph, then set small increments of the
desired brake percent to measure deceleration using the encoders.

We ran tests using small increments in throttle percent to characterize
the acceleration-to-throttle percent relationship. We measured the
car's acceleration using the encoders mounted on the rear wheels.

To test the odometry for tracking position, we used Rviz2 to keep track
of the car's orientation and position relative to where it started in
the world frame. Since we could not test the throttle indoors, we
pushed the car while in neutral with the maximum steering angle to
travel in a circle to ensure the odometry returned to where we started.
