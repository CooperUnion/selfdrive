\section{Background}

\subsection{Organization}

The team follows a vertically integrated structure, divided into three
disciplines consisting of three to five student groups, each led by a
subteam lead who facilitates weekly meetings. The subteams correspond
to abstract systems on the car: Hardware (Electrical + Mechanical),
Firmware, and Algorithms. In addition to weekly subteam meetings, we
hold weekly team meetings to keep everyone in the loop of new
developments and bi-weekly sprint meetings to establish tasks for our
bi-weekly test days.

The hardware subteam designs and assembles all expansion hats for the
Cooper Common Microcontroller Nodes (CCMNs), a custom
\texttt{\#coopermade} breakout board around the ESP32S3
microcontroller, to facilitate the needs of the Firmware subteam. In
addition to PCBs, the subteam is responsible for maintaining the power
distribution system of the car and designing mechanical components to
retrofit our vehicle.

Our Firmware subteam writes all our safety critical actuation code and
custom tooling to supplement software development. They aim to write
modular firmware, allowing for the fast extension of CCMNs attached to
hardware hats to specific hardware applications.

The algorithms subteam handles the perception and navigation of our
car. They create \& tune controllers, detect objects, perform
localization based on camera and encoder data, and generate paths given
all our component outputs.

\subsection{Design Assumptions \& Design Process}

Our design process revolves around ensuring our systems are safe and
easy to test in the cramped surroundings of the East Village in
downtown Manhattan. We try to make our systems as modular as possible
by breaking down every Autonomy Lab project into its various
components, allowing for overlap between different projects we may take
on. We host all our work in a publically available in a Git monorepo,
which is licensed under copyleft licenses to ensure free access to our
work to everyone. Maintaining a Git monorepo allows for easy
collaboration on components, simplifies code review, and lets us
accurately track changes to our work.

Due to Cooper's small size, it's easy for us to keep tabs on one
another to ensure things get done, but outside of meetings, we also
communicate using a Cooper-hosted instance of Matrix. We also believe
that maintaining good documentation is essential to avoid amassing
technical debt, so we keep a mix of public-facing documentation in our
monorepo and institution-specific documentation that does not make
sense to share outside Cooper.
